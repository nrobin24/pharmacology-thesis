% Chapter 1

\chapter{Introduction} % Main chapter title

\label{Chapter1} % Change X to a consecutive number; for referencing this chapter elsewhere, use \ref{ChapterX}

\lhead{Chapter 1. \emph{Introduction}} % Change X to a consecutive number; this is for the header on each page - perhaps a shortened title

%----------------------------------------------------------------------------------------
%	1.1 - Background
%----------------------------------------------------------------------------------------

\section{Background}

%-----------------------------------
%	1.1.1 - Stem Cell Signaling and Disease
%-----------------------------------
\subsection{Stem Cell Signaling and Disease}

Stem cells are cells that can divide both symmetrically and asymmetrically. This means that stem cells can renew their own population, as well as produce differentiated cell types of various lineages \cite{Basics_2009}. Human embryonic stem cells (hESCs) reside within developing human embryos and are pluripotent. This means that hESCs can differentiate into any of three germ layers, and ultimately are capable of yielding any cell type found in the body. In addition to embryonic stem cells, there are stem cells which reside in certain tissues during adulthood, which can alternatively be referred to as progenitor cells. Adult stem and progenitor cells are lineage restricted (multipotent), as opposed to pluripotent. For example, adult neural progenitor cells may differentiate to become mature neurons, astrocytes, or glia, but are restricted from becoming vascular cells. Adult stem cells are responisble for maintaining homeostasis in tissues. This is especially the case in areas with rapid cell turnover, such as in hair folicles and the intestinal lining \cite{Potten_1998, Morris_2004}.

Stem cell populations in both developing and adult organisms are typically found within complex microenvironments called niches \cite{Scadden_2006}. Stem cell niches contain stem cells as well as support cells. Support cells within niches can be responsible for providing nutrients to stem cells, and also can provide signals guiding either differentiation or proliferation. The decisions that stem cells undergo regarding when to proliferate, when to differentiate, and which lineage to specify upon differentiation are the result of a complex mixture of intrinsic and extrinsic signals.

Although we lack a complete understanding of the signals that influence these decisions in stem cells, it is clear that when signaling becomes misregulated there can be deleterious effects. Changes to stem cell proliferation and differentiation, caused by genetic or environmental factors, contribute to the pathology of a number of diseases. In cases where a cell type that is necessary for maintaining healthy tissue is no longer produced, degeneraiton occurs. Alternately, tumors occur when stem cell proliferation goes unchecked. It is clear then, that we must futher our understanding of the signals that influence stem cell decisions in order to develop therapies for these diseases. Moreover, if we possess pharmacological tools to direct stem cell behavior we may be able to develop methods to repair diseased tissue within a patient.

%-----------------------------------
%	Wnt in Cancer Tumorigenesis
%-----------------------------------

\subsection{Wnt in Cancer Tumorigenesis}

Misregulation of the Wnt pathway has been linked to various types of tumors. The cancer stem cell hypothesis holds that the same pathways regulate self-renewal in stem cells and cancer cells, and that there are stem cells within the tumor that possess indefinite potential for self-renewal. [Tannishtha Reya. Nature 2001]

Different tumors have found to have Wnt signaling component. Colorectal cancer is linked to activating APC mutations. Melanoma through XXX. Breast cancer thru XXX. Willms tumor is linked to WTX. Glioblastoma is linked through ??. Also renal cell, non small cell lung cancer, ec.

It is not always over-activation that causes cancer. In melanoma patients activating the pathway is beneficial. Altho this may be time dependent.

%-----------------------------------
%	Wnt in hESCs
%-----------------------------------

\subsection{Wnt in hESCs}

There is a controversy as to the specific role of canonical signaling.

The existence of this controversy means that we should study the Wnt pathway in hESCs more closely and try to learn more about the specific mechanisms.

The controversy also speaks to the complexity of hESC culture, where results can often conflict with mouse ESCs, and also multiple signaling pathways are highly dependant upon one another.

QUOTE Thus, heterogeneity with respect to endogenous Wnt signalling underlies much of the inefficiency in directing hESCs towards specific cell types. The relatively uniform differentiation potential of the Wnt(high) and Wnt(low) hESCs leads to faster and more efficient derivation of targeted cell types from these populations.

http://www.ncbi.nlm.nih.gov/pubmed/22990866

Also Ex-Vivo Directed Differentiation

%-----------------------------------
%	Wnt in hESCs
%-----------------------------------

\subsection{Wnt in Adult Neurogenesis}

It was discovered in X that there are populations of adult neural progenitor cells that reside in the brain and contribute new functioning neurons throughout adulthood. Wnt was discovered to be a key regulator of adult neurogenesis by Fred Gage's lab.

Secreted negative regulators of Wnt signaling increase throughout adulthood, and this correlates with reduction in neurogenesis.

Mutations in genes that regulate  Wnt signaling in the brain (DISC1, DIXDC1) are linked with schizophrenia and major depression.

Thorough discussion can be found in simvastatin paper

%-----------------------------------
%	Pharmacologic regulators of Wnt/ß-Catenin signaling
%-----------------------------------

\subsection{Pharmacologic regulators of Wnt/ß-Catenin signaling}

Currently there are no clinically approved therapies that specifically target Wnt signaling. [Zimmerman et al.] This is in spite of the fact that misregulation of the pathway has a well known role in cancer and degenerative diseases. 

Lithium is in clinical use and part of its mechanism may be due to Wnt signaling. This is talked about in my Simva paper.

Difficulties include: progenitor cell populations like in the intestine and in germinal regions of the brain require Wnt signaling, so inhibiting the pathway in order to block growth of a cancer with activated Wnt may be too harmful to other tissues. [@Watanabe_Dai_2011] Another problem is that GSK3, which is a kinase and thus relatively easier to target with small molecules, is also integral to AKT signaling and performs an additional function in synthesizing glycogen. [@TakahashiYanaga2013191]

There are now multiple compounds under pre-clinical investigation.

%----------------------------------------------------------------------------------------
%	SECTION 2 - Approach
%----------------------------------------------------------------------------------------

\section{Approach}

%-----------------------------------
%	Discovering Small Molecule Regulators of Cell Signaling
%-----------------------------------

\subsection{Discovering Small Molecule Regulators of Cell Signaling}

In the research presented, I looked to discover small molecule regulators of the Wnt/ß-catenin pathway. I was particularly interested in small molecules because of their applicability to both basic research and clinical drug development. As opposed to genetic approaches such as siRNAs and shRNAs to induce a gain or loss of function in cell signaling, small molecules can be included in an experimental paradigm with relatively little overhead and higher relative predictability of effect [CITATION]. This effect carries over when administering small molecules to animals, which tends to be more efficacious than genetic materials. This can be due to cell permeability and bioavailability [CITATION]. Finally, small molecules have a long proven track record in treating human diseases.

Consistent with my overall goals, small molecules are available in the form of arrayed libraries. This is important in that it allows for high-throughput screens to be performed by automated equipment, rather than being tested individually by hand. These contents of the libraries can be produced from curated lists, such as the library of human experienced compounds in the riluzole in simvastatin papers. Libraries can also be produced combinatorially around a scaffold in order to target a specific motif such as the ATP binding pocket. This type of library is seen in the Wiki paper and also in the JW screen.

Finally, I prioritized the discovery of small molecule regulators of Wnt/ß-Catenin signaling because the small molecules that I reported on served as deliverables in and of themselves alongside the journal articles. This means that future projects have a reliable physical starting place from which to conduct experiments. The molecules I report on are commercially available, and in some cases are already approved for human use. This means that the time it takes for another group to begin experiments based off of my findings will be minimized, and ongoing studies of these molecules such as in clinical trials will be able to draw inference from my projects.

%-----------------------------------
%	Understanding Biological Mechanisms through Cell Signaling Events
%-----------------------------------

\subsection{Understanding Biological Mechanisms through Cell Signaling Events}

The projects that I participated in attempted to further our understanding of  cell behavior. In each case I used Wnt/ß-Catenin signaling as an entry point from which to address biological mechanisms that govern behavior. This entailed selectively monitoring and perturbing Wnt signaling. I monitored Wnt signaling while using established methods to perturb other pathways, in order to find where multiple pathways may cross-talk with one another. To determine the role of Wnt in homeostatic and disease processes, I perturbed Wnt signaling while monitoring behaviors such as cell death, proliferation, induction of differentiation, and lineage specification.

I chose to focus on Wnt/ß-catenin signaling in part because of its established role in diseases including melanoma and various neurodegenerative disorders. This helped to guide the direction in which I proceeded through projects, as I was able to place my experimental results within context of the body of knowledge about the specific disease.

In spite of what is known about how misregulated Wnt/ß-catenin signaling can precipitate disease states, there are relatively few therapies targeting this signaling pathway. This served as another reason that I focused on this pathway.

%-----------------------------------
%	Exploiting Cell Signaling to Inform Novel Therapeutics
%-----------------------------------

\subsection{Exploiting Cell Signaling to Inform Novel Therapeutics}

The idea of targeting pathways in disease has been around for a while

There were certain successes, particularly in cancer, see Gleevec

However cancer is fast moving and changing, see Vemurafinib

A new idea is to target cancer stem cells with Wnt

Why? Wnt has less functional redundancy than ERK/MAPK

An aim of my work was to identify opportunities where perturbation of cell signaling in stem and progenitor cells may inform the development of a rational therapeutic.

This approach worked in my projects where:

with ril and simva we already knew something about targets

with WIKI we were able to connect to known mechanism

later on with BRAF, Wnt was able to synergize w Vemurafenib

To date, the field of translational life sciences has made major headway towards development of therapies that utilize pharmacologic manipulation of stem and progenitor cells. 